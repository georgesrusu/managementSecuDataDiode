\documentclass[a4paper,10pt]{article}
\usepackage[utf8]{inputenc}
\usepackage[english]{babel}
\usepackage{indentfirst}
\usepackage{listings}
\usepackage{graphicx}
\usepackage{blindtext}
\usepackage{enumitem}
\usepackage{hyperref}
\usepackage{multirow}
\usepackage[top=2.5cm,bottom=2.5cm,left=2.5cm,right=2.5cm]{geometry}
\pagestyle{headings}
\title{Ftp server in a datadiode}
\author{Rusu George, Boulif Ilias, Orinx Cédric}
\date{\today}

\begin{document}
\maketitle

\newpage
\section{Risk Analysis}
\subsection{Assets and Vulnerabilities}
\subsubsection{Physical Assets}
\subsubsection{Logical Assets}
\subsubsection{Persons}
\subsubsection{Intangible Goods}

\subsection{Threat Sources}

\paragraph{Nature:} Natural disasters such as an earthquake or a storm could be a source of threat that would affect the building of the company.

\paragraph{Employees:} All employees that don't have the necessary knowledge to interact with the data diode must be taken into account of the analysis.

\paragraph{Administrators:} Since that they have all access and priviledge to the data diode, the adminstrators are possible threat source.

\paragraph{Script Kiddies:} They can a be a potentiel source due to the fact that the system is connected to the internet.

\paragraph{Skilled Hacker:} The information protected by the system can be a target for different reasons like selling the the data or getting information for rival company. Since the system is supposed to be well protected, attacks will require some skills.

\paragraph{Unauthorized user}

\paragraph{Malware:} 

\subsection{Vulnerabilities}

\paragraph{} The following table will give a set of possible vulnerability that we have to take in account to secure the most possible the data diaode. This table will give the \textbf{ID} of vulnerability, his \textbf{source}, in reference to the point 1.2, which part of the inforamtion security it will \textbf{affect} (\textit{confidentiality}, \textit{avaibility}, \textit{integrety}), and a short \textbf{description} of the vulnerability.

\begin{table}[!h]
	\centering
	\begin{tabular}{|c|p{2.5cm}|c|p{2.5cm}|p{6.5cm}|}
		\hline
		\textbf{ID}&  \textbf{Vulnerability} & \textbf{Source(s)} & \textbf{Affecting} & \textbf{Description} \\
		\hline
		1 & Power outage & Nature & Avaibility &  \\
		\hline
		2 & Fire & Nature & Avaibility &  \\
		\hline
		3 & DDOS attack & Skilled Hacker & Avaibility &  \\
		\hline
		4 & Data diode hardware
		failure & Nature & Avaibility & \\
		\hline
		5 & Zero Day Attack & Skilled Hacker & Avaibility, Confidentiality, Integrety & \\
		\hline
		6 & Data diode hardware degradation & Unauthorized user & Avaibility, Confidentiality, Integrety & \\
		\hline
		7 &  &  &  & \\
		\hline
		8 &  &  &  & \\
		\hline
		9 &  &  &  & \\
		\hline
		10 &  &  &  & \\
		\hline
		11 &  &  &  & \\
		\hline
	\end{tabular}
	\caption{Vulnerabilities}
\end{table}

\subsubsection{Vulnerabilities Affecting Physical Assets}
\subsubsection{Vulnerabilities Affecting Logical Assets}
\subsubsection{Vulnerabilities Affecting Persons}
\subsubsection{Vulnerabilities Affecting Intangible Goods}

\subsection{Risks and Countermeasures}

\begin{table}[!h]
	\centering
	\begin{tabular}{|c|p{2.5cm}|c|p{10cm}|}
		\hline
		\textbf{ID}& \textbf{Vulnerability} &\textbf{Impact} & \textbf{Description}                 \\
		\hline
		1 & Power outage & Medium  &  The data diode will not work anymore untill the power will be reasablish. \\
		\hline
		2 & Fire & High  &  The data diode can suffer significant damage wich could lead to the replacment of the hardware.\\
		\hline
		3 & DDOS attack & High &\\
		\hline
		4 & Data diode hardware failure  & Low &\\
		\hline
		5 & Zero Day attack & High &\\
		\hline
		6 & Data diode harware degradation & High &\\
		\hline
	\end{tabular}
	\caption{Impact}
\end{table}

\begin{table}[!h]
	\centering
	\begin{tabular}{|c|p{2,5cm}|c|p{10cm}|}
		\hline
		\textbf{ID}& \textbf{Vulnerability} & \textbf{Likelihood} & \textbf{Description}                 \\
		\hline
		1 & Power outage & Low &   \\
		\hline
		2 & Fire & Low  &  \\
		\hline
		3 & DDOS attack & Low & \\
		\hline
		4 & Data diode hardware failure  & Medium & \\
		\hline
		5 & Zero Day attack & Medium & \\
		\hline
		6 & data diode hardware degradetion & Medium & \\
		\hline
	\end{tabular}
	\caption{Likelihood}
\end{table}

\paragraph{} After having analyzed the impact and the likelihood of each vulnerabilities, we can define for each of them a risk level using the Table \ref{table:RL} to adapt our countermesure accordingly.

\begin{table}[!h]
	\centering
	\begin{tabular}{|l|l|l|l|}
		\hline
		\multicolumn{4}{|c|}{\textbf{Risk Level}} \\
		\hline
		\textbf{Likelihood} & \multicolumn{3}{|c|}{\textbf{Impact}}  \\ \hline
		 & \textbf{Low} & \textbf{Medium} & \textbf{High} \\ \hline
		\textbf{Low}& Low & Low & Low \\ \hline
		\textbf{Medium} & Low & Medium & Medium \\ \hline
		\textbf{High} & Low & Medium & High \\ \hline
	\end{tabular}
	\caption{Risk Level}
	\label{table:RL}
\end{table}


\begin{table}[!h]
	\centering
	\begin{tabular}{|c|p{2,5cm}|p{2.5cm}|p{6.5cm}|c|c|c|}
		\hline
		\textbf{ID}& \textbf{Vulnerability} & \textbf{Source(s)} & \textbf{Countermeasure(s)} & \textbf{I} & \textbf{L} & \textbf{Risk Level}          \\
		\hline
		1 & Power outage  & Nature & & M & L & Low \\
		\hline
		2 & Fire & Nature & & H & L & Low \\
		\hline
		3 & DDOS attack & Skilled hacker & & H & L & Low \\
		\hline
		4 & Data diode hardware failure & Nature & & L & L & Low \\
		\hline
		5 & Zero Day attack & Skilled hacker & & H & M & Medium \\
		\hline
		6 & Data diode hardware degradation & Unothorized user & & H & M & Medium  \\
		\hline
		7 & & & & & & \\
		\hline
		8 & & & & & & \\
		\hline
		9 & & & & & & \\
		\hline
	\end{tabular}
	\caption{Countermeasure}
\end{table}


\end{document}